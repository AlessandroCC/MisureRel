\chapter{Misure con oscilloscopio digitale su amplificatore operazionale in configurazione catena chiusa}
\label{chap:seconda_prova}

\section*{Obbiettivo}
L'obbiettivo della terza esperienza di laboratorio è quello di valutare il guadagno e la frequenza di taglio di un amplificatore operazionale in configurazione a \emph{catena chiusa} (closed-loop).

\section{L'amplificatore operazionale µA741}
L'amplificatore operazionale (OP-AMP) è un dispositivo attivo, dotato cioè di alimentazione, in grado  di amplificare i segnali in ingresso. Durante l'esperienza è stato usato l'OP-AMP µA471, provvisto dei seguenti pin:
\begin{itemize}
    \item V\textsubscript{+} morsetto non invertente
    \item V\textsubscript{-} morsetto invertente
    \item V\textsubscript{o} pin d'uscita
    \item +V\textsubscript{cc} e -V\textsubscript{cc} pin per l'almentazione in continua
    \item due pin per la compensazione dell'offset
\end{itemize}

